\documentclass[12pt, twoside, openright]{ctexbook}

\usepackage[a4paper, top=2.5cm, bottom=2.5cm, left=2.2cm, right=2cm]{geometry}

\setmainfont{Times New Roman}
\setCJKmainfont{KaiTi}

\setCJKfamilyfont{zhkai}[AutoFakeSlant=0.3, AutoFakeBold = {2.17}]{KaiTi}
\renewcommand*{\kaishu}{\CJKfamily{zhkai}}

\setCJKfamilyfont{zhsong}[AutoFakeSlant=0.3, AutoFakeBold = {2.17}]{SimSun}
\renewcommand*{\songti}{\CJKfamily{zhsong}}

\newcommand{\pinyinzh}{\fontsize{21pt}{1.1\baselineskip}\selectfont} % 三号,1.5倍行距
\newcommand{\wenzizh}{\fontsize{44pt}{1.1\baselineskip}\selectfont} % 三号,1.5倍行距 
\newcommand{\zhushizh}{\fontsize{28pt}{1.1\baselineskip}\selectfont} % 三号,1.5倍行距

\usepackage{longtable}
\usepackage{tabu}
\usepackage{setspace}
\usepackage{makecell}

%----------------------------页边空白调整-------------------------------
\def\marginset#1#2{                      % 页边设置 \marginset{left}{top}
\setlength{\oddsidemargin}{#1}         % 左边(书内侧)装订预留空白距离
\iffalse                   % 如果考虑左侧(书内侧)的边注区则改为\iftrue
\reversemarginpar
\addtolength{\oddsidemargin}{\marginparsep}
\addtolength{\oddsidemargin}{\marginparwidth}
\fi
\setlength{\evensidemargin}{0mm}       % 置0
\iffalse                   % 如果考虑右侧(书外侧)的边注区则改为\iftrue
\addtolength{\evensidemargin}{\marginparsep}
\addtolength{\evensidemargin}{\marginparwidth}
\fi
% \paperwidth = h + \oddsidemargin+\textwidth+\evensidemargin + h
\setlength{\hoffset}{\paperwidth}
\addtolength{\hoffset}{-\oddsidemargin}
\addtolength{\hoffset}{-\textwidth}
\addtolength{\hoffset}{-\evensidemargin}
\setlength{\hoffset}{0.5\hoffset}
\addtolength{\hoffset}{-1in}           % h = \hoffset + 1in
\setlength{\voffset}{-1in}             % 0 = \voffset + 1in
\setlength{\topmargin}{\paperheight}
\addtolength{\topmargin}{-\headheight}
\addtolength{\topmargin}{-\headsep}
\addtolength{\topmargin}{-\textheight}
\addtolength{\topmargin}{-\footskip}
\addtolength{\topmargin}{#2}           % 上边预留装订空白距离
\setlength{\topmargin}{0.5\topmargin}
}
% 调整页边空白使内容居中,两参数分别为纸的左边和上边预留装订空白距离
\marginset{2mm}{12mm}

% \setlength\LTleft{-1in}
% \setlength\LTright{-1in plus 1 fill}

% Change the intercolumn space
% \setlength{\tabcolsep}{2pt}

\title{\kaishu \bfseries \wenzizh 千\\{~}\\字\\{~}\\文}
% \author{\pinyinzh 整理:\\\pinyinzh 排版:}
\author{OuTlK@QQ.CoM}
\date{\pinyinzh 2019年8月12日}

\begin{document}

\frontmatter
\begin{spacing}{1.19}
\begin{titlepage}
    \maketitle
\end{titlepage}
\thispagestyle{empty}
\centering
\fontsize{14pt}{\baselineskip}\selectfont
\input{"qzw_all.tex"}
\cleardoublepage

% \newpage
\mainmatter
% \pagestyle{plain}
% \setcount{page}{1}
\begin{longtabu} to \textwidth {p{1.55cm}<{\centering}p{1.55cm}<{\centering}p{1.55cm}<{\centering}p{1.55cm}<{\centering}p{.1cm}p{1.55cm}<{\centering}p{1.55cm}<{\centering}p{1.55cm}<{\centering}p{1.55cm}<{\centering}}
% \begin{longtabu} to \textwidth {ccccccccc}
    % \caption{Add caption}
    % \begin{tabular}{ccccccccc}
    % {\pinyinzh A} & {\pinyinzh B} & {C} & {D} & {E} & {F} & {G} & {H} &  \\
    % {A} & {B} & {} & {} & {} & {} & {} & {} &  \\
    % \multicolumn{9}{c}{\sanhao A} \\
    % \input{"qzw.tex"}
    % \end{tabular}%
    % \label{tab:addlabel}%
    \centering
    % {\pinyinzh zhuang} & {\pinyinzh zhuang} & {\pinyinzh zhuang} & {\pinyinzh zhuang} & &{\pinyinzh zhuang} & {\pinyinzh zhuang} & {\pinyinzh zhuang} & {\pinyinzh zhuang} \\
    % {\sanhao 天} & {\sanhao 地} & {\sanhao 玄} & {\sanhao 黄} & &{\sanhao 宇} & {\sanhao 宙} & {\sanhao 洪} & {\sanhao 荒} \\
    % \multicolumn{9}{p{\textwidth}}{\pinyinzh 天是青黑色的,地是黄色的,宇宙形成于混沌蒙昧的状态中。太阳正了又斜,月亮圆了又缺,星辰布满在无边的太空中。}\\
    % tiān & dì & xuán & huáng & & yǔ & zhòu & hóng & huāng \\
    \input{"qzw.tex"}
\end{longtabu}
\end{spacing}

\end{document}
